\documentclass{article}
\usepackage[utf8]{inputenc}

\begin{document}

This study presents an investigation into the effectiveness of genetic algorithms (GA) for configuring optical logic gates using photonic crystal fibers. We aimed to demonstrate that GA could find optimal configurations even in complex systems with multiple cores, as opposed to traditional geometric constraints.

\section*{Results and Discussion}
The results from our simulations were highly promising. Specifically, we successfully achieved good contrast in the output of the OR gate, reaching a value close to 0.541210 dB despite challenges associated with increased complexity in systems involving four or more cores (text content). This highlights the robustness and flexibility of GA when applied to these types of problems.

\begin{table}[H!]
    \centering
    \caption{TABELA 1: Mostra claramente como diferentes combinações de entradas (I0, I1, I2, I3) resultam em saídas lógicas OR.}
    ...
\end{table}

Our approach contrasts with previous work by Martins et al., who utilized triangular equilateral arrangements for configuring OR and AND gates but faced limitations due to specific geometric constraints (text content). By employing GA, our study overcomes such restrictions, offering flexibility in exploring various configurations without being confined to particular shapes.

A critical analysis of the results reveals several key points. Firstly, the effectiveness of combining photonic crystal fibers with genetic algorithms for configuring optical logic gates is evident. This approach allows us to explore a wide range of possible configurations systematically (text content). Additionally, our study contributes robust methodologies and replicable outcomes that align well with intended purposes.

However, limitations in the current methodology must be addressed. Specifically, certain GA steps or computational configurations have not been fully elucidated, leaving room for improvement. Further research could focus on exploring additional optimization algorithms or hybrid techniques alongside GA to enhance efficiency in finding ideal configurations (text content).

Furthermore, extending these configurations to more complex logic gates and practical information technology and communication contexts represents a valuable direction for future work.

\section*{Conclusions}
To summarize the key findings from this study, we have demonstrated that genetic algorithms can effectively configure optical OR gates using photonic crystal fibers. The ability to achieve high contrast outputs even in systems with multiple cores showcases the potential of GA in overcoming traditional geometric constraints associated with configuring logic gates (text content). These results contribute significantly to the broader field of optical computing and photonic crystal fiber research, offering new avenues for technological advancements in gate design and implementation using photonic crystals.

\end{document}